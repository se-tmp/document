\documentclass[11pt, conference]{IEEEtran}
\IEEEoverridecommandlockouts

\usepackage{dirtree}
\usepackage{cite}
\usepackage{amsmath,amssymb,amsfonts}
\usepackage{algorithmic}
\usepackage{graphicx}
\usepackage{textcomp}
\usepackage{xcolor}
\usepackage{kotex}
\usepackage{booktabs}
\usepackage{tabularx}
\usepackage{supertabular,booktabs}
\usepackage{adjustbox}
\usepackage{enumitem}
\usepackage{romannum}
\usepackage{makecell}
\usepackage{multirow}
\usepackage{hyperref}
\usepackage{graphics}
\usepackage{subfigure}
\usepackage{float}
\usepackage{listings}
\usepackage[edges]{forest}
\hbadness=99999  % or any number >=10000
\vbadness=99999  % or any number >=10000
\hfuzz=20pt
\def\BibTeX{{\rm B\kern-.05em{\sc i\kern-.025em b}\kern-.08em T\kern-.1667em\lower.7ex\hbox{E}\kern-.125emX}}
\begin{document}

% function
% add new image with full width
% usage:
% \addImageFull{
%     imgs/specification/login_page.png
% }{
%     description to image
% }
\newcommand{\addImageFull}[2]{
  \begin{figure}[ht]
    \begin{center}
      \includegraphics[width=8cm]{#1}
      \caption{#2} % description to image
      \renewcommand{\thefigure}{\thesubsection.\arabic{figure}}
    \end{center}
  \end{figure}
}

% function
% add new image with specific size
% usage:
% \addImageSize{
%     5cm
% }{
%     imgs/specification/login_page.png
% }{
%     description to image
% }
\newcommand{\addImageSize}[3]{
  \begin{figure}[ht]
    \begin{center}
      \includegraphics[width={#1}]{#2}
      \caption{#3} % description to image
      \renewcommand{\thefigure}{\thesubsection.\arabic{figure}}
    \end{center}
  \end{figure}
}

\setcounter{figure}{0}

\title{TempoMate Project\\
  \small{Seamless home automation based on posture/position detection\\}
}

\makeatletter
\newcommand{\linebreakand}{
  \end{@IEEEauthorhalign}
  \hfill\mbox{}\par
  \mbox{}\hfill\begin{@IEEEauthorhalign}
}
\makeatother

\author{
  \IEEEauthorblockN{Jo Taesik}
  \IEEEauthorblockA{\textit{dept. of Information Systems} \\
    \textit{Hanyang University}\\
    Seoul, Korea\\
    r4pidstart@hanyang.ac.kr}
  \and
  \IEEEauthorblockN{Kwon Jongin}
  \IEEEauthorblockA{\textit{dept. of Information Systems} \\
    \textit{Hanyang University}\\
    Seoul, Korea \\
    whddlswhdaud@naver.com}
  \and
  \IEEEauthorblockN{Bae Hyojeong}
  \IEEEauthorblockA{\textit{dept. of Information Systems} \\
    \textit{Hanyang University}\\
    Seoul, Korea \\
    bhj09270@hanyang.ac.kr}
  \linebreakand % <------------- \and with a line-break
  \IEEEauthorblockN{Lee Hyunsuk}
  \IEEEauthorblockA{\textit{dept. of Information Systems} \\
    \textit{Hanyang University}\\
    Seoul, Korea \\
    leehyunsuk2000@gmail.com}
  \and
  \IEEEauthorblockN{Nan Haixu}
  \IEEEauthorblockA{\textit{dept. of Information Systems} \\
    \textit{Hanyang University}\\
    China, Guangzhou \\
    what-is-my-id@naver.com}
}

\maketitle
\begin{abstract}
  \textit{With the recent surge in fascination with home automation, numerous companies are investigating strategies to facilitate the unified management of connected devices. A few of these techniques comprise verbal requests to a service for a particular action or the activation of a specific action when motion is sensed by designated sensors. Nevertheless, these approaches possess limitations that necessitate users to execute certain requests or actions that they would not typically perform to direct their devices. Devices also do not enable the full comprehension of the user's intentions. These restrictions do not meet the requirements of users who seek to construct home automation that can naturally recognize their intentions and respond correspondingly. Our proposal is to offer a service permitting users to initiate particular actions by way of genuine, intentional behavior that feels natural. This service offers a platform to assimilate and manage devices via the Matter protocol. On this platform, users can determine which actions are activated based on users interactions with certain objects in specific locations. By using IP cameras connected to the platform, OpenPose, Library for pose estimation assesses the user's posture, labelling it as sitting, lying down, standing, and more. By recognizing the posture of the user and specific objects, pre-defined actions are triggered. With this service, User can advance beyond traditional home automation to create a system that operates by comprehending user's intents with greater precision.\\}
  % 최근 홈 오토메이션에 대한 관심이 높아짐에 따라, 여러 기업들은 연결된 기기를 통합적으로 간편하게 제어하기 위한 방법을 연구하고 있습니다. 음성으로 서비스를 호출하여 특정 액션을 요청한다던지, 특정 센서에 움직임이 감지되면 특정 액션이 트리거되는 등의 방법이 존재합니다. 그러나 이러한 방법들에는 한계가 존재합니다. 사용자로 하여금 기기를 제어하기 위해서가 아니라면 하지 않았을,  특정한 호출이나 동작을 강제합니다. 또한 사용자가 진정으로 무엇을 의도하는지 알 수 없다는 점도 있습니다. 이러한 한계는 자연스럽게 내 의도를 파악하여 그에 맞는 행동을 하는 홈 오토메이션을 구축하길 원하는 사용자들의 니즈를 충족시킬 수 없습니다. 그래서 우리는 사용자들이 부자연스럽지 않은, 의도가 담긴 자연스러운 동작을 통해 특정 액션을 트리거할 수 있는 서비스를 제안합니다. 이 서비스는 Matter 프로토콜을 이용해 기기들을 통합하고, 관리할 수 있는 플랫폼을 제공합니다. 이 플랫폼에서, 유저는 어느 위치의 어떤 사물에서 어떤 동작을 취하는지에 따라 어떤 액션이 트리거될 지를 설정할 수 있습니다. 이 플랫폼에 연결된 카메라를 이용해, openpose가 유저의 자세를 추정합니다. 이때 앉거나, 눕거나, 일어나는 등으로 유저의 자세를 분류합니다. 이런 방법으로 지정된 사물에서 설정해놓은 자세를 취하는 것을 인식하면 미리 지정된 액션을 트리거합니다. 이 서비스를 이용해 유저는 기존의 홈 오토메이션보다 한 단계 앞서 작동하고, 유저의 의도를 더욱 더 잘 파악하여 동작하는 홈 오토메이션을 완성할 수 있습니다. 
\end{abstract}

\begin{IEEEkeywords}
  home automation, IOT, Matter, pose estimation, OpenPose \\\\\\\\
\end{IEEEkeywords}

\large{Role Assignments}
\begin{table}[H]
  \center
  \begin{tabular}{m{1.4cm} m{1.5cm} m{4cm}}
    \toprule
    Roles               & Name                   & Task description \& etc.                                                                                                                                                                                                                                                \\
    \midrule
    User                & Bae Hyojeong           & He expects that he will be able to use them to make his life easier by operating IOT devices, However, He is hesitant because of the negative reviews from people who have already used them, or he is afraid that it will be too difficult to install and set them up. \\\\
    Customer            & Kwon Jongin            & Created a product that can be controlled using IOT, but consumers rarely use the feature because it's not as convenient as expected. He is looking for ways to make his product more convenient to use in order to get consumers to use product.                        \\\\
    Software developer  & Lee Hyunsuk, Nan Haixu & Designs and implements a solution to a given problem. Able to troubleshoot problems that may arise during development and complete assigned tasks within a given timeframe.                                                                                             \\\\
    Development manager & Jo Taesik              & Breaking down a given task into solvable problems and distributing them appropriately among team members. They also coordinate schedules to ensure that tasks are completed within a given timeframe.                                                                   \\
    \bottomrule
  \end{tabular}
\end{table}
\newpage

\section{\Large{Introduction}}
% \begin{enumerate}[label=\arabic*]
%%%%%%%%%%%%%%%%%%%%%%%%%%%%%%%%%%%%%%%%%%%%
\subsection {\large{Motivation}}
The IOT market has been growing rapidly in recent years. According to IOT Analytics, the IOT device market, which was valued at \$120 billion in 2019, is growing at a staggering rate, reaching \$100 billion in the first quarter of 2023. This market is also expected to grow at a CAGR of 20\% in the future. According to the same organization, there are currently an estimated 14.4 billion connected IOT devices, which is also expected to grow at a CAGR of 16\%. \\

\begin{figure}[ht]
    \includegraphics[width=8cm]{imgs/introduction/iot_market_growing.png}
    \caption{Growing IOT market}
    \renewcommand{\thefigure}{\thesubsection.\arabic{figure}}
\end{figure}

In response to this market growth, many electronics manufacturers have begun to include IOT-related technologies in their products, ranging from passive technologies that allow you to control your device through an application on your phone, to technologies that allow you to control multiple devices with a single device in conjunction with devices such as smart speakers, to more advanced technologies such as air quality sensors and motion recognition sensors that allow you to operate your device automatically without human command. \\

Consumer interest in IOT technologies is on the rise, and according to OpenSurvey, the number of consumers who own a home appliance with these technologies has grown from 29.5\% in 2022 to 48.3\% in 2023, an increase of 18.8 percentage points over the previous year.\cite{iot-market-size} \\\\

\begin{figure}[ht]
    \includegraphics[width=8cm]{imgs/introduction/reason_why_not_use_iot.png}
    \caption{Reasons for not using IOT devices}
    \renewcommand{\thefigure}{\thesubsection.\arabic{figure}}
\end{figure}

However, consumers who have tried these technologies are often disappointed with devices that don't deliver on their pre-purchase expectations and stop using them. There are a few main reasons why this happens. According to OpenSurvey, these are the top reasons consumers have IOT devices but don't use them. \cite{iot-opensurvey}\\
\begin{enumerate}
    \item They are more comfortable operating them manually (59.1\%)\\
    \item They have different apps/devices for each device they own and don't want to use multiple devices at the same time (28.6\%)\\
    \item They have tried the apps/devices but don't like the usability (14.6\%)\\
    \item The apps/devices don't work well with their devices (11.0\%)\\
\end{enumerate}

We propose a platform to address the necessary issues for consumers to utilize IoT technologies. Our integrated IoT platform employs posture recognition to activate desired functions.\\\\

\newpage
%%%%%%%%%%%%%%%%%%%%%%%%%%%%%%%%%%%%%%%%%%%%
\subsection {\large{Problem Statement}}
\begin{enumerate}[label=\alph*]
    \item Uncomfortable Automation\\
          There is a gap between what users think home automation should be and what current platforms offer. Users assume that machines will interpret their intentions and complete tasks without any input, but the reality is quite the opposite. The machines can only perform specific procedures that have already been entered, and they lack the ability to determine when those procedures should be executed. Currently, the sole means of inducing a machine to execute a task is through summoning a voice assistant with a designated command and directing it to execute a specific procedure. This hindrance fosters the belief that it is more convenient to manually operate a device than to avail oneself of automation. Consequently, users will avail themselves of IOT features only under highly restricted circumstances.\\

    \item Complexity of Use\\
          Currently, IOT devices from manufacturers and platforms can only be controlled through voice recognition. However, this method is too limited for users who desire a fully-fledged smart home. Third-party platforms exist for these users, which provide more detailed and advanced features. Although these platforms are not officially supported by the manufacturers, users who encounter problems must solve them on their own. Connecting devices to the platform and building the server are required tasks. Additionally, these platforms these platforms describe their behavior in code, which means users have to get used to it. These factors increase complexity and may present a barrier to entry for the users.\\

    \item Security Concerns\\
          Devices that use a manufacturer's platform can pose security risks by sending all user and device information to the manufacturer's servers. This compromise exposes sensitive personal information and lifestyle details to potential exploitation. Additionally, it opens up the possibility for malicious actors to manipulate and misuse devices in the home remotely.\\

    \item Disjointed communication methods\\
          Numerous communication methods are utilized in IoT devices these days, encompassing traditional WiFi and Bluetooth as well as UWB, RFID, Zigbee, Z-WAVE, XBee, LoRa, SigFox, and many others. Device manufacturers have selected one or more communication methods to construct their devices, resulting in consumers being limited to devices that use a specific communication method based on the hub they utilize.\\

          \begin{figure}[ht]
              \includegraphics[width=7.5cm]{imgs/introduction/iot-protocols.png}
              \cite{iot-protocols}
              \caption{IOT protocols}
              \renewcommand{\thefigure}{\thesubsection.\arabic{figure}}
          \end{figure}

    \item Fragmented Platforms\\
          IOT device manufacturers maintain distinctive platforms and require navigating their platforms and hubs for using a specific device of a manufacturer. This limitation restricts the full functionality of devices to the manufacturer's designated platform, rendering interoperability difficult. Some manufacturer hubs do not allow devices from other manufacturers to connect, causing consumers to lean towards one manufacturer for their smart home devices and discouraging the use of products from different manufacturers at the same time. It also creates challenges when replacing a product line that is not made by a particular manufacturer, thus limiting consumer options.

\end{enumerate}
%%%%%%%%%%%%%%%%%%%%%%%%%%%%%%%%%%%%%%%%%%%%
\subsection {\large{Related Software}}
\begin{enumerate}[label=\alph*]
    \item Google Home\\
          \addImageSize{5cm}{imgs/introduction/home.png}{Google Home}\\
          This is a smart home platform offered by Google. It is compatible with a variety of devices, including lights, thermometers, speakers, and more. Additionally, it works with the Google Assistant, which allows you to check the status of your devices or control them. The same can be done through the accompanying application. Routines tailored to specific circumstances can be created and activated via voice commands. Easily add family members to share routines and jointly control devices. In addition to setting things up in the application, user can automate the details using YAML scripts.\\\\

    \item Apple Home\\
          \addImageSize{5cm}{imgs/introduction/homekit.png}{Apple Homekit}\\
          This is a smart home platform offered by Apple that is compatible with a variety of devices including lights, thermometers, and speakers, among others. It works with the Siri, which allows you to check the status of your devices or control them. The same can be done through the accompanying application. One can conveniently monitor device status via Control Center or widgets on other Apple devices. An Apple TV can also serve as a hub and automatically regulate devices based on circumstance, including weather, motion, or humidity. However, an Apple device is required.\\\\

    \item Samsung Smartthings\\
          \addImageSize{6cm}{imgs/introduction/smartthings.png}{Samsung Smartthings}\\
          This is a smart home platform offered by Samsung. This platform is compatible with a variety of devices, including lights, thermometers, and speakers. Users can access and manage device status via Bixby or the application. To take full advantage of the IOT capabilities of their Samsung home appliances, users need this platform. User can utilize their Samsung devices as sensors. Unlike platforms offered by other manufacturers, Smartthings is an open platform, allowing for connection of even unsanctioned products through various actions. The hub permits connection of third-party sensors and setup of automation routines.\\\\

    \item LG ThinQ\\
          \addImageSize{7cm}{imgs/introduction/thinq.png}{LG ThinQ}\\
          This is a smart home platform designed by LG that is necessary for full utilization of the IOT features of LG appliances. Users can receive notifications when their appliances finish their tasks, diagnose problems with their appliances, and schedule after-sales services. They can also integrate with Apple HomeKit to monitor or control their HomeKit-connected devices using this application.\\\\


    \item Home Assistant\\
          \addImageSize{7cm}{imgs/introduction/homeassistant.png}{Home Assistant}\\
          This is the sole mainstream smart home platform founded on open-source principles. The software can be installed and run on various hardware, including Raspberry Pi or Odroid. Devices with Home Assistant behave as hubs once the software is installed.\\

          \begin{figure}[ht]
              \begin{center}
                  \raggedleft
                  \includegraphics[width=8cm]{imgs/introduction/ha_script_example.png}
                  \caption{Home Assistant script example}
                  \renewcommand{\thefigure}{\thesubsection.\arabic{figure}}
              \end{center}
          \end{figure}

          It supports a diverse range of protocols and easily connects to devices unsupported by it through multiple actions. It is more difficult to configure than other platforms, but after mastering it, scripts can be utilized to access any data from linked devices and automate actions in numerous ways.\\


\end{enumerate}
\section{\Large{Requirement Analysis}}
\begin{enumerate}[label=\arabic*.]
    \item {\large{Login}}\\

    \begin{enumerate}[label*={\arabic*.},ref=\theenumi.\arabic*]
    \setlength{\itemindent}{0.5cm}
        \item The user must log in to their own account, which contains their personal information.\\
        
        \item This application should be customizable for homeowners, so user’s personal information need to be stored in his/her own account.\\
        
        \item If the user manages multiple devices, the information is required for connecting to several devices.\\
        
        % \item User can add and remove devices he/she manages through QR code(It is also possible without the QR code).\\
        
        \item For the login process, user inputs their ID and password within the app.\\
        
        \item Alternatively, users can utilize an external ID, like a Google or Apple ID, to conveniently sign in without having to register.\\
    
        \item ID must include at least 5 English or English and numbers both.\\
    
        \item Password must include at least 4 alphabetic uppercase letters, special characters, and numbers.\\

        \item When the user sign up, he/she sets a username.\\

        \item A username must more than 3 characters.\\

        \item And a username is 
    \end{enumerate}

    \item {\large{Logs and Statistics}}\\
    \begin{enumerate}[label*={\arabic*.},ref=\theenumi.\arabic*]
    \setlength{\itemindent}{0.5cm}
        \item The application records user activities, including events such as trigger activation time, routine name, and user's location. \\
        
        \item It maintains device interaction logs, tracking device status changes, usage patterns, and scheduled events or automations triggered by the user. \\
        
        
        \item User behavior, preferences, and usage patterns are monitored to enhance this service. \\
        
        \item To achieve this, it stores a history of device status data, allowing users to review past device activities.\\
    \end{enumerate}


    \item {\large{Dashboard}}\\
    \begin{enumerate}[label*={\arabic*.}]
        \item {\large{Device Connection}}\\
            \begin{enumerate}[label*={\arabic*.},ref=\theenumi.\arabic*]
            \setlength{\itemindent}{0.5cm}
            \item This application provide users with the ability to connect devices, including device pairing and disconnection.\\
            \item Implement a QR code-based device connection method, allowing users to easily pair devices by scanning QR codes.\\
            \item Offer an alternative method for connecting to devices without the use of QR codes, providing flexibility in device pairing.\\
        \end{enumerate}
        
        \item {\large{Device Status Check}}\\
         \begin{enumerate}[label*={\arabic*.},ref=\theenumi.\arabic*]
            \setlength{\itemindent}{0.5cm}
            \item This application implements a system to log and monitor the status of connected devices, including health assessments and connectivity information.\\
            \item It should create and maintain records of status check activities, recording when these checks were performed, the specific devices involved, and the results indicating device operational status or reported issues.\\
        \end{enumerate}
        
        \item {\large{Device Control}}\\
         \begin{enumerate}[label*={\arabic*.},ref=\theenumi.\arabic*]
            \setlength{\itemindent}{0.5cm}
            \item  Implement the ability for users to manually control devices, including actions like turning devices on or off, adjusting settings, and triggering automation sequences.\\
            \item Establish a logging system to track these control actions, capturing the date and time of each action, specifying the devices affected, and identifying the responsible user for each action.\\
            \item Provide users with the capability to view the live feed or recorded content from connected devices like home security cameras.\\
        \end{enumerate}
    \end{enumerate}

    \item {\large{Routine Managemant}}\\
    \begin{enumerate}[label*={\arabic*.}]
        \item {\large{Trigger Settings}}\\
            \begin{enumerate}[label*={\alph*.},ref=\theenumi.\arabic*]
            \setlength{\itemindent}{0.5cm}
            
                \item {\large{Sound Trigger}}\\
                \begin{enumerate}[label*={\arabic*.},ref=\theenumi.\arabic*]
                \setlength{\itemindent}{0.5cm}
                    \item Users can set conditions based on the intensity of sound to trigger accessories or scenes.\\
                    \item For example, when a loud sound is detected, it can trigger the application to send notifications or initiate camera recording events.\\
                    \item When it comes to speakers that can recognize voice, users can also employ voice recognition to determine what the user has said as a condition to trigger activities. \\
                    \item For example, users can perform a series of preset actions after a word.\\
                \end{enumerate}
                
                \item {\large{Sensor Trigger}}\\
                \begin{enumerate}[label*={\arabic*.},ref=\theenumi.\arabic*]
                \setlength{\itemindent}{0.5cm}
                    \item It can trigger related functions according to the information of various sensors. For example, connecting smoke and gas sensors.\\
                    \item Smoke and gas sensors can detect dangerous conditions, such as fire or gas leakage. Once the problem is detected, the system will trigger alarm, inform users and take necessary measures.\\
                    \item It can also connect sensors like ultrasonic sensors to detect the presence of objects in certain areas. If any movement is detected when the user is away, it will send a notification to the user.\\
                \end{enumerate}
                
                \item {\large{Time Trigger}}\\
                \begin{enumerate}[label*={\arabic*.},ref=\theenumi.\arabic*]
                \setlength{\itemindent}{0.5cm}
                    \item Accessories or scenes can be triggered based on a fixed time of day, certain days, or based on sunrise and sunset. \\
                    \item For example, turn on the curtains and music at 6 a.m. every morning, or turn off all lights 15 minutes after sunset.\\
                \end{enumerate}
                
                \item {\large{Posture Recognition Trigger}}\\
                \begin{enumerate}[label*={\arabic*.},ref=\theenumi.\arabic*]
                \setlength{\itemindent}{0.5cm}
                    \item Capturing images or videos through the camera and utilizing OpenPose for body posture recognition also allows for the identification of body movements.\\
                    \item Different actions can trigger various functions based on the recognized movements.\\
                \end{enumerate}

                \item {\large{Position Trigger}}\\
                \begin{enumerate}[label*={\arabic*.},ref=\theenumi.\arabic*]
                \setlength{\itemindent}{0.5cm}
                    \item The user's location can be determined based on the location information of the user's mobile phone, and corresponding functions can be triggered based on the user's location.\\ 
                    \item For example, when the user returns home, they automatically turn on the lights and automatically start the air conditioner.\\
            \end{enumerate}
    
        \end{enumerate}
        \item {\large{Behavior Settings}}\\

        \begin{enumerate}[label*={\arabic*.},ref=\theenumi.\arabic*]
        \setlength{\itemindent}{0.5cm}
            \item The user shall be able to set the behavior in accordance with the trigger if it is triggered from a specific setting, such as sound, sensor, time, posture, etc.\\

            \item Actions are required to turn the lights off and on, to control the intensity of the light, and to control the time.\\

            \item To set these actions, a button is required to add the desired action
            \\
            
            \item Add items such as devices, time delays, and broadcasts to be controlled by adding actions.
            \\
            
            \item A button to delete is also required if the action is not required
            \\
            
            \item You can set up more than one action with a single trigger and must support collaboration between multiple devices.
            \\

            \item Each trigger and each user shall be able to set the name of the routine and select enable/disable the routine.
            \\
    \end{enumerate}
\end{enumerate}

    \item {\large{Application Settings}}\\
    \begin{enumerate}[label*={\arabic*.},ref=\theenumi.\arabic*]
    \setlength{\itemindent}{0.5cm}
        \item Each user is required with a button to modify so that the username can be set.\\

        \item Users should implement a logout button so that they can logout when they want to change their device or account.\\

        \item A push alarm on/off button must be implemented to determine whether or not to receive an alarm.\\

        \item A push alarm on/off button must be implemented to determine whether or not to receive an alarm.\\
        
        \item Each time each user logs in, backup and restore routines should be implemented without the need to create a new routine\\
        
        \item A button to delete all settings should be implemented so that each person can initialize them.\\

    \end{enumerate}
\end{enumerate}
\section{\Large{Development Environment}}
\begin{enumerate}[label=\arabic*]
    \item {\large{Task distribution}}
    \begin{table}[H]
    \center
    \begin{tabular}{m{1.4cm} m{1.5cm} m{4cm}}
    \toprule
    Roles & Name & Task description \& etc.\\
    \midrule
    \\
    Project management & Jo Taesik & The project manager assumes numerous responsibilities throughout the planning and execution stages of a project. Specifically, they establish project objectives, delegate weekly tasks and define corresponding roles, monitor team members' work output and progress, and make pertinent adaptations and assignments to ensure that the project abides by established deadlines. Additionally, they collect all team members' project documentation and edit it to fashion a comprehensive manuscript. \\\\
    \bottomrule
    \end{tabular}
    \end{table}
    
    \begin{table}[H]
    \center
    \begin{tabular}{m{1.4cm} m{1.5cm} m{4cm}}
    UI/UX design & Jo Taesik & UI/UX designers are accountable for designing the UI and UX. They establish the fundamental layout and functionality of the application, and complete the visual design. They design workflows and pathways for the users and optimize the accessibility of frequently-used features. The objective is to help users use the finalized product effectively and efficiently. \\\\
    \bottomrule
    \end{tabular}
    \end{table}

    \begin{table}[H]
    \center
    \begin{tabular}{m{1.4cm} m{1.5cm} m{4cm}}
    App frontend & Kwon Jongin, Jo Taesik & Front-end developers take the UI/UX designs created by designers and transform them into code that users can directly view and interact with on the application. Through the use of multiple frameworks and technologies, they write code that guarantees a design’s intended appearance on any device. Additionally, they utilize APIs and sockets to communicate with the backend, exchanging and displaying relevant data on the screen. \\\\
    \bottomrule
    \end{tabular}
    \end{table}

    \begin{table}[H]
    \center
    \begin{tabular}{m{1.4cm} m{1.5cm} m{4cm}}
    App backend & Nan Haixu, Bae Hyojeong & Application backend developers develop and manage the server-side part of the application. They design and develop databases to manage the necessary data and process user data effectively. They develop APIs to communicate with clients and servers so they can interact. They manage data by encrypting it to keep user data secure. \\\\
    \bottomrule
    \end{tabular}
    \end{table}

    \begin{table}[H]
    \center
    \begin{tabular}{m{1.4cm} m{1.5cm} m{4cm}}
    Computer vision & Lee Hyunsuk & Computer vision developers create programs for image and video processing and interpretation using libraries like OpenCV and OpenPose. They design and implement algorithms to extract desired information from visual data. \\\\
    \bottomrule
    \end{tabular}
    \end{table}
    
\end{enumerate}

% \newpage

\section{\Large{Specifications}}
\begin{enumerate}[label=\arabic*.]

%%%%%%%%%%%%%%%%%%%%%%%%%%%%%%%%%%%%%%%
    \item {\large{Example}}
    \begin{enumerate}[label*={\arabic*.},ref=\theenumi.\arabic*]
    \setlength{\itemindent}{0.5cm}
        \item
            \begin{table}[H]
            \center
                \begin{tabular}{m{1.4cm} m{5.5cm}}
                \toprule
                \# of Req. & Description\\
                \midrule
                Req 1.1. & The user must log in to their own account, which contains their personal information.\\\\
                Req 1.5. & Alternatively, users can utilize an external ID, like a Google or Apple ID, to conveniently sign in without having to register.\\\\
                \bottomrule
                \end{tabular}
            \end{table}
            
                \begin{figure}[ht]
                    \begin{center}
                    \includegraphics[width=3.5cm]{imgs/specification/login_page.png}
                    \caption{Login page}
                    \renewcommand{\thefigure}{\thesubsection.\arabic{figure}}
                    \end{center}
                \end{figure}
            There are two buttons on login page, 'Continue with Google' and 'Continue with Apple'. When the user click the button, login is success and move to the main page.\\\\

        \item
            \begin{table}[H]
            \center
                \begin{tabular}{m{1.4cm} m{5.5cm}}
                \toprule
                \# of Req. & Description\\
                \midrule
                Req 1.1. & The user must log in to their own account, which contains their personal information.\\\\
                \bottomrule
                \end{tabular}
            \end{table}
            
            \begin{center}
                \begin{figure}[ht]
                    \begin{center}
                    \includegraphics[width=3.5cm]{imgs/specification/login_page.png}
                    \caption{Login page}
                    \renewcommand{\thefigure}{\thesubsection.\arabic{figure}}
                    \end{center}
                \end{figure}
            \end{center}
            There are two buttons on login page, 'Continue with Google' and 'Continue with Apple'. When the user click the button, login is success and move to the main page.\\\\
    \end{enumerate}
%%%%%%%%%%%%%%%%%%%%%%%%%%%%%%%%%%%%%%%
    
    \item {\large{Login Page}}
    \begin{enumerate}[label=\alph*]
        \item There are two buttons on login page, 'Continue with Google' and 'Continue with Apple'. When the user click the button, login is success and move to the main page.
        \item On the main page, there is 'plus' button which makes the page move to 'add Device' page. The user can add Matter devices through QR code on this page. There is also the button, 'Connect without the QR code'. 
    \end{enumerate}
    \item {\large{Log Page}}
    \begin{enumerate}[label=\alph*]
        \item At the Bottom of the page, there is the menu named 'Log'. When the user click the 'Log' button, He/She can see their routine logs. This application records and stores information such as the user's location, posture, and sensors by time period 
    \end{enumerate}
    \item {\large{Routine Page}}\\
    Here you can view all plans that the user has set. You can click on them to view the status, or click the start button to start them manually. If you need to create a new plan yourself, you can also click the plus sign in the lower right corner to customize the plan by setting triggers and activities.\\
    \begin{enumerate}[label=\alph*]
        \item Create new routine\\
            Click the plus sign in the lower right corner of the page to create your own routine. Enter the name of the desired routine in the pop-up window and click "Add" to enter the routine setting page.\\

            
        \item Routine Setting Page\\
            Click the set routine page to adjust the set routine. In the settings page, you can modify the name of the routine, share the routine, and delete the routine. On the right side you can select the switch for this routine. Below is the setting of trigger type. Different conditions can be selected according to different triggers. After selecting, click the check mark in the upper right corner to save.\\
            \begin{enumerate}
                \item  Location Trigger Settings\\
                    To do.\\
                \item  Posture Trigger Settings\\
                    After clicking on the pose trigger settings, first select the camera you want to use, and then you can select the action you want to trigger. There is now a choice of three recognized postures, sitting, standing and lying down. You can choose to perform the following actions after the camera recognizes which posture it is. Such as turning on a light or turning on a switch. The actions performed can also be set by yourself.\\
                \item  Voice Assistant Trigger Settings\\
                    After clicking the voice assistant trigger setting, you can choose the voice command when you want to execute the command. Different triggers can be triggered based on the recognized voice command.\\
                \item  Schedule Trigger Settings\\
                    After clicking on the scheduled trigger setting, you can select the time to be set in the prompt box below, and the trigger will be automatically executed after this time.\\
            \end{enumerate}
            
        \item behavior routine
        \begin{enumerate}
            \item It was implemented to adjust the intensity of the light through the light-on motion bulk sldier, and the switch on/off function was implemented\\
            

            \item In the routine item, a todo list was implemented for each trigger, such as construct detect and schedule, and actions to be performed for each todo were implemented. Switch off time delay
        \end{enumerate}
        \item setting
            \begin{enumerate}
            \item The on/off button that alerts you when the routine is started by a particular trigger has implemented a button that can be turned on and off according to touch, with the 'push alarm on route starts' written in the notification-shaped column\\

            \item Each user created a Backup/Restore column to back up and import routines, and Backup routes implemented it with the download shape and Restore routes with the upload shape\\

            \item username in a pencil shape so that intuitively see that it can be modified, and created a logout button at the bottom and implemented it so that easily change my account\\

            \item Implemented 'Delete all settings' with trash bin shape to delete all settings
    \end{enumerate}
\end{enumerate}
\end{enumerate}
\newpage
\section{\Large{Architecture Design \& Implementation}}

\begin{enumerate}[label=\arabic*]

\newcommand{\addImage}[2]{
        \begin{figure}[!htb]
            \begin{center}
                \includegraphics[width=5.5cm]{#1}
                \caption{#2} % description to image
                \renewcommand{\thefigure}{\thesubsection.\arabic{figure}}
            \end{center}
        \end{figure}
        
    }
    \item {\large{Overall Architecture}}\\
        % todo

    \item {\large{Directory Organization}}\\
    % add directory structure table
    % simply explain which directory is for what
    
    \begin{enumerate}[label=\alph*]
        \item Frontend\\
        \item Backend
\begin{table}[H]
    \centering
    \begin{tabular}{m{3cm} m{4.5cm}}
        \toprule
        Directory & File Name \\
        \midrule
        se-tmp/ backend/ & src/ main \newline pom.xml  \\
        \midrule
        se-tmp/ backend/ src/ \newline main/ & java/ com/ tempomate/ \newline resources/ \\
        \midrule
        se-tmp/ backend/ src/ \newline main/ resource/ & application.properties \\
        \midrule
        se-tmp/ backend/ src/ \newline main/ java/ com/ \newline tempomate/ & controller/ \newline exception/ \newline mapper/ \newline pojo/ \newline service/ \newline SebackendApplication.java\\
        \midrule
        se-tmp/ backend/ src/ \newline main/ java/ com/ \newline tempomate/  controller/ & ActionController.java \newline DeviceController.java \newline LogController.java \newline RoutineController.java \newline TriggerController.java \newline UserController.java \\
        \midrule
        se-tmp/ backend/ src/ \newline main/ java/ com/ \newline tempomate/ exception/ & GlobalExceptionHandler.java \\
        \midrule
        se-tmp/ backend/ src/ \newline main/ java/ com/ \newline tempomate/ mapper/ & ActionMapper.java \newline DeviceMapper.java \newline LogMapper.java \newline RoutineMapper.java \newline TriggerMapper.java \newline UserMapper.java  \\
        \midrule
        se-tmp/ backend/ src/ \newline main/ java/ com/ \newline tempomate/ pojo/ & entity/ \newline Result.java\\
        \midrule
        se-tmp/ backend/ src/ \newline main/ java/ com/ \newline tempomate/ pojo/ \newline entity/ & action/ \newline trigger/ \newline Device.java \newline Log.java \newline Routine.java \newline User.java\\
        \midrule
        se-tmp/ backend/ src/ \newline main/ java/ com/ \newline tempomate/ pojo/ \newline entity/ action/ & ActDevice.java \newline ActTime.java \newline Action.java \\
        \midrule
        se-tmp/ backend/ src/ \newline main/ java/ com/ \newline tempomate/ pojo/ \newline entity/ trigger/ & TriAssistant.java \newline TriLocation.java \newline TriPosture.java \newline TriTime.java \\
        \midrule
        se-tmp/ backend/ src/ \newline main/ java/ com/ \newline tempomate/ service/ & impl/ \newline UserService.java\\
        \midrule
        se-tmp/ backend/ src/ \newline main/ java/ com/ \newline tempomate/service/ impl/ & UserServiceImpl.java \\
        \bottomrule
    \end{tabular}
\end{table}
        \item Computer vision\\
    \end{enumerate}

    \item {\large{Architecture Implementation}}\\
    \begin{enumerate}[label=\alph*]
        \item Frontend\\
        \begin{enumerate}
            \item Purpose \\
            TempoMate's backend serves as the central nervous system, orchestrating seamless communication between users, frontend interfaces, and the intricate web of routines, triggers, and actions within the application. Its overarching purpose is to empower users to effortlessly manage their smart home environment, utilizing real-time video streaming via WebRTC to detect user positions and postures, specifically when they are resting in bed. By understanding and interpreting these scenarios, the backend triggers predefined routines, seamlessly executing actions such as turning off room lights. Additionally, the backend administers essential user functionalities, including registration, login, logout, and nickname changes, ensuring a secure and personalized experience for each user. \\\\
            \item Functionality \\
            TempoMate offers an intuitive interface that empowers users to effortlessly harness the sophisticated capabilities orchestrated by the backend. The frontend serves as the user's gateway to a personalized and responsive smart home environment.\\

            User Account Management:
            The frontend streamlines the user account management process, providing a seamless experience for tasks such as registration, login, logout, and nickname changes. Through clear and user-friendly interfaces, individuals can easily establish and customize their profiles within the TempoMate ecosystem.\\

            Routines Management:
            Users have full control over defining their smart home environment through the frontend's management of routines. They can intuitively add, delete, and share routines, tailoring their automation preferences with simplicity. The frontend visually represents these routines, allowing users to grasp and modify their smart home orchestration effortlessly.\\

            Action Customization:
            The frontend facilitates the addition and deletion of actions, empowering users to specify precise responses to detected video scenarios. Through the frontend interface, users can fine-tune the behavior of their smart home, ensuring that it aligns precisely with their preferences and needs.\\

            Trigger Activation:
            Managing triggers becomes an accessible task through the frontend, enabling users to activate and expand smart home routines effectively. The frontend provides a clear representation of triggers, allowing users to understand and adjust the conditions that initiate specific actions within their smart home environment.\\

            Real-time Video Streams:
            The frontend seamlessly integrates real-time video streams captured via WebRTC, providing users with live insights into their smart home environment. Through the frontend, users can monitor and interact with the captured video, enhancing their situational awareness and control.\\

            In essence, the frontend of TempoMate not only simplifies user management but also empowers users to shape and interact with a smart home environment that is both adaptive and responsive, thanks to the sophisticated functionalities orchestrated by the backend. \\\\
            
            \item Location of source code \\
            
            : https://github.com/se-tmp/frontend \\\\
            
            \item Class Component \\
                \item[-] ActionPage.kt \\
                The provided code includes several Jetpack Compose functions for building the user interface of an Android application. These functions handle different aspects of the app's functionality, such as selecting action types, choosing devices, and configuring actions for device control. The code demonstrates the use of Jetpack Compose features, creating a well-structured and maintainable codebase for managing the app's UI and user interactions.\\
                \item[-] AddActionSelectDevice.kt \\
                The AddActionSelectDevice composable class renders a grid of selectable devices retrieved from a DevicesRepository. Utilizing Jetpack Compose's LazyVerticalGrid, the devices are efficiently displayed in a two-column grid with appropriate spacing. Each device is represented by a DeviceCard with an associated icon and name. When a user selects a device by clicking on its card, the chosen device is stored in the IotViewModel, and the navigation is triggered to go back (onNavigateUp). This composable efficiently handles the presentation of selectable devices, enhancing the user experience for adding actions related to specific devices in the smart home environment\\
                \item[-] Dashboard.kt \\
                The Dashboard composable function renders a grid of devices, each represented by a DeviceCard, providing an overview of the smart home environment. The list of devices and their states is periodically updated by querying repositories (DevicesRepository and DevicesStateRepository). The devices are displayed in a two-column grid using LazyVerticalGrid, with appropriate spacing and padding. Each DeviceCard includes details such as the device name, icon, switch state, and online status. Users can interact with the cards by adjusting the device state or clicking on them to view more details. The composable uses coroutines to handle periodic updates of device information. This Dashboard composable serves as a central hub for users to monitor and control their smart home devices efficiently.\\
                \item[-] DeviceCard.kt \\
                The DeviceCard composable is a versatile component designed to represent various types of cards within a smart home application. It supports different card types (DeviceCardType), such as those for the dashboard, routines, and device selection. The card layout includes an icon, device name, and status information, with specific buttons or elements based on the card type. The card may feature an on/off switch (Switch) for the dashboard and routine cards, and it can handle adjustments with the provided onAdjust callback. Additionally, the card may display an "Offline" status for devices in the dashboard that are not currently online.\\
                \item[-] DevicePage.kt \\
                DevicePage serves as a comprehensive view for managing and interacting with a specific device, encompassing its status, actionable buttons, and a visual representation of the device. This composable enhances the user experience by providing a consolidated interface for device-related actions.\\
                \item[-] IotViewModel.kt \\
                The ViewModel initializes with user data repository, Gson for JSON serialization/deserialization, and loads initial data when created.\\\\
                Loading Initial Data : Retrieves initially saved routines from the user data repository, parsing and updating the UI state accordingly.\\

                Saving Routines : Asynchronously saves the provided list of routines to the user data repository in JSON format.\\

                Setting User ID : Updates the UI state with the user ID.\\

                Setting Device : Updates the UI state with the currently selected device.\\

                Appending Action on Routine : Adds the current action to the actions list of the current routine in the UI state.\\

                Appending Routine : Adds or modifies a routine in the UI state based on whether it already exists.\\

                Deleting Action on Routine by Index : Removes an action from the actions list of the current routine based on the provided index.\\

                Setting Current Action : Updates the UI state with the current action.\\

                Setting Actions : Updates the UI state with a list of actions.\\

                Setting Trigger Type of Current Routine : Updates the UI state with the trigger type of the current routine.\\

                Setting Trigger of Current Routine : Updates the UI state with the trigger of the current routine.\\

                Setting Triggers : Updates the UI state with a list of triggers.\\

                Setting Routines : Updates the UI state with a list of routines.\\

                Setting Current Routine : Updates the UI state with the current routine.\\
                \item[-] LoginPage.kt \\
               The provided code comprises three Jetpack Compose functions for user authentication in an IoT application. LoginPage displays a login page, allowing users to sign in with email/password or Google. SignIn handles email/password authentication, while SignUp manages user registration. These functions utilize FirebaseAuth for authentication and seamlessly navigate to the Devices screen upon successful login or signup, enhancing the overall user authentication flow.\\
               \item[-] RoutinePage.kt \\
               The provided code includes Jetpack Compose functions for creating various UI components related to an IoT application. TitleBar, TitleButtons, TriggerTypes, Trigger, AddActionButton, Action-ControlDevice, Action, Actions, RoutinePage, PointerCircle, and ResizableRectangle contribute to building the user interface for managing routines and actions. These components cover aspects such as displaying routine titles, handling triggers, showing action lists, and providing interactive elements for creating and editing routines.\\
               \item[-] Routines.kt \\
               Routines is a Composable function that uses Jetpack Compose to display a grid of routine cards. It takes an IotViewModel as a parameter to access the UI state, a function onClickCard to be executed when a routine card is clicked, and an optional modifier for styling. The routine data is collected from the UI state using viewModel.uiState.collectAsState().value.routines.\\

               The grid is implemented using LazyVerticalGrid with fixed columns, and each routine is represented by a DeviceCard composable. The DeviceCard includes information such as the routine's name, icon, and on/off status. When a routine card is clicked, it updates the current routine in the view model and triggers the onClickCard function.\\
               \item[-] Settings.kt \\
               Settings is a Composable function that represents a screen displaying user settings, including a title bar, notification settings, backup and restore options, version information, and a delete all settings option. It utilizes Jetpack Compose for UI development and Firebase Authentication (auth: FirebaseAuth) for handling user authentication. The screen layout is organized using various Compose components such as Column, Row, Text, Icon, Switch, and Button.\\
               \item[-] LotScreen.kt \\
               The App Composable function represents the main structure of your IoT application. It uses Jetpack Compose for UI development and is composed of various screens corresponding to different IotScreen enum values. The app includes features such as user authentication, a dashboard displaying devices, routines management, logs, settings, and various actions. The App Composable orchestrates the navigation flow, state management, and UI structure of your IoT application, providing a cohesive user experience across different screens.\\
        \end{enumerate}

        \item {\large{Database Implementation}}\\
        % \begin{enumerate}[label=\alph*]
        \addImage{
            imgs/architecture_design_and_implementation/database-erd.png
        }{
            Database Diagram
        }
        This is the database that our architecture is currently using. This database consists of 11 tables. By using the structure of these tables, the data generated by the project can be stored.\\
            % \item Frontend\\
            % % \begin{enumerate}
            % % \end{enumerate}
        % \end{enumerate}
        
        \item Backend\\
        \begin{enumerate}
            \item Purpose \\
            TempoMate's backend serves as the central nervous system, orchestrating seamless communication between users, frontend interfaces, and the intricate web of routines, triggers, and actions within the application. Its overarching purpose is to empower users to effortlessly manage their smart home environment, utilizing real-time video streaming via WebRTC to detect user positions and postures, specifically when they are resting in bed. By understanding and interpreting these scenarios, the backend triggers predefined routines, seamlessly executing actions such as turning off room lights. Additionally, the backend administers essential user functionalities, including registration, login, logout, and nickname changes, ensuring a secure and personalized experience for each user. \\\\
            \item Functionality \\
            The backend of TempoMate enhances user experience and enables various functionalities for smart home automation. It efficiently handles user account management tasks, including user registration, authentication (login/logout), and user nickname changes. The core of the app lies in its capability to manage routines, allowing users to define their smart home environment easily through features such as adding, deleting, and sharing routines. Similarly, the backend manages the addition and deletion of actions, enabling users to define specific responses to detected video scenarios. Additionally, the backend facilitates the effective activation of smart home routines by managing the addition and expansion of triggers. Lastly, the backend seamlessly processes real-time video streams captured via WebRTC, forwarding them to the frontend and AI components to perform their respective functions. In summary, the multifaceted functionalities of the backend not only ensure a robust user management system but also establish a responsive and adaptive smart home environment based on AI and routines. \\\\
            \item Location of source code \\
            : www.github.com/se-tmp/backend \\\\
            \item Class Component \\
                \item[-] pom.xml: This file is associated with the Maven build management tool commonly used in Java projects. It contains project configuration information and is written in XML format. It is used to define project settings, dependencies, plugins, and other build-related information. This Maven POM (Project Object Model) file defines and configures a Spring Boot-based web application. The project has dependencies on `spring-boot-starter-web` for initiating a Spring Boot web application, `mybatis-spring-boot-starter` for integrating MyBatis with Spring Boot to support database access, `mysql-connector-j` for connecting to a MySQL database, `lombok` for increased code conciseness, and `spring-boot-starter-test` for supporting testing in a Spring Boot application. The build is configured using the `spring-boot-maven-plugin`, excluding Lombok from compilation. The project inherits from the `spring-boot-starter-parent` with a version of 2.7.17 and is set to Java version 11. This architecture is centered around building a Spring Boot web application with MyBatis integration for efficient database access. \\
                \item[-] src/main/resources/application.properties : This configuration file defines settings for a Spring Boot application, specifying the context path as "/api" and providing connection details for a MySQL database.\\
                \item[-] src/main/java/com.tempomate/controller: This is the folder that contains controllers which handle user input and return the results to the user. \\
                \item[-] ActionController: This class handles API operations related to actions using POST, GET, and DELETE mappings. The "/actDevice\_add" and "/actTime\_add" endpoints use POST requests to respectively add device actions and time delay actions. The "/get\_all\_action/{userId}" endpoint retrieves all actions stored in the database using a GET request, where {userId} is the user identifier. The "/delete/{id}" endpoint deletes the action corresponding to the provided ID from the database using a DELETE request.\\
                \item[-] DeviceController: This class handles various mappings related to devices, including POST, DELETE, GET, and PUT. The "/add" endpoint uses POST to add a device to the database. The "/delete/{id}" endpoint, using DELETE, removes the device with the specified ID from the database. The "/get\_all\_device/{userId}" endpoint, with GET, returns all devices associated with the given user ID. The "/rename\_device" endpoint, using PUT and receiving a new device name in the request body, updates the device name. Lastly, the "/change\_status" endpoint, through PUT, receives a request to change the device's switch within the range of 0 to 100. \\
                \item[-] LogController: This class handles logging, and the "/user/{userId}" endpoint, through a GET mapping, returns a list of all logs for the user corresponding to the user ID in the endpoint. \\
                \item[-] RoutineController: This class handles various mappings related to routines, encompassing POST, DELETE, GET, and PUT methods. The "/add" endpoint is used to create a new routine using a POST request. The "/delete/{id}" endpoint, through a DELETE mapping, processes the deletion of the routine associated with the provided ID. The "/share/{id}" endpoint provides an API for sharing a specific routine. The "/get\_all\_routine/{userId}" endpoint, utilizing the GET method, returns a list of all routines associated with the specified `{userId}` value. The "/rename\_routine" endpoint, receiving a request for a new routine name, updates the routine's name using a PUT request. The "/backup/{id}" endpoint is utilized to back up a specific routine. This endpoint performs the logic to back up the routine, and upon successful completion of the backup process, it returns a response accordingly. The "/restore" endpoint is employed for restoring a routine using previously backed-up information. This endpoint executes the restoration logic, and upon successful restoration of the routine, it returns a response indicating the successful outcome. Finally, the "/change\_status" endpoint, taking the routine's ID and on/off status in the request body, updates the routine's push notification status using a PUT request. \\
                \item[-] TriggerController: This class uses POST mappings to add triggers and GET mappings to retrieve them. The endpoints "/loc\_add," "/pos\_add," "/assi\_add," and "/time\_add" are used to respectively add location triggers, posture triggers, assistant triggers, and time triggers through POST requests. The "/get" endpoint receives triggerType and triggerId in the request and returns the corresponding trigger using a POST mapping. Additionally, conditional statements are used to fetch different triggers based on the trigger type of the routine. For each trigger type, it retrieves the appropriate trigger (TriLocation, TriPosture, TriAssistant, TriTime) from the database. \\
                \item[-] UserController: This class handles user-related information using POST and PUT mappings. The "/signup" endpoint deals with user registration using a POST request. The "/login" endpoint is a POST API that receives the user's ID and password in the request for performing login. The "/rename-nickname" endpoint updates the user's nickname by receiving a new nickname through a PUT request. The "/push\_setting" endpoint manages the user's push notification status using a PUT request. The "/logout" endpoint handles user's logout session.\\
                \item[-] src/main/java/com.tempomate/exception/ \par GlobalExceptionHandler: This code defines a global exception handler in a Spring Boot application.\\
                \item[-] src/main/java/com.tempomate/mapper : This is the folder that is employed to define and execute SQL queries for interacting with a database. \\
                \item[-] ActionMapper: This interface interacts with the database to provide functionality related to actions. The `addAction`, `addActDevice`, and `addActTime` methods are responsible for adding action, device action, and time delay action, respectively, to the database. The `getActDevice` and `getActTime` methods retrieve device actions and time delay actions associated with a user and routine. Lastly, the `deleteAction` method removes all action data from the database. \\
                \item[-] DeviceMapper: This interface handles database operations for the `Device` entity. It includes methods to add a new device, retrieve all devices associated with a specific user, delete a device based on its ID, rename a device, and update the switch status of a device. \\
                \item[-] LogMapper : This interface handles database retrieval operations related to the `Log` entity. The `get\_all\_log` method retrieves all logs associated with a specific user ID from the database.\\
                \item[-] RoutineMapper: This interface manages database operations related to the `Routine` entity. It includes methods for adding a new routine, deleting a routine based on a specific ID, retrieving all routines associated with a particular user, renaming a routine, and updating the active/inactive status of a routine. \\
                \item[-] TriggerMapper: This interface is responsible for handling database operations related to triggers in the context of routines. This includes methods for adding triggers of different types (location, posture, assistant, and time) and retrieving specific trigger information based on the trigger ID associated with a routine. \\
                \item[-] UserMapper: This interface handles database operations related to user management. It includes methods for retrieving a list of users (primarily used for testing purposes), inserting a new user, performing user login authentication, changing a user's nickname, and updating the push notification settings for a user. \\
                \item[-] src/main/java/com.tempomate/pojo/entity : This is the folder that contains all entity files. \\
                \item[-] ActDevice: This code defines a ‘ActDevice' entity representing device action information, utilizing Lombok annotations for concise code. The entity includes fields for an ID (‘id’), routine ID (‘routineId’), device ID (‘deviceId’), switch status (‘switchStatus’), text (‘text’), and action order ('order1'), allowing for a simplified representation of these attributes.\\
                \item[-] Action: This code defines a ‘Action' entity representing action information, utilizing Lombok annotations for concise code. The entity includes fields for an ID (‘id’), routine ID (‘routineId’), action type (‘actionType’), and action ID (‘actionId’), allowing for a simplified representation of these attributes. \\
                \item[-] ActTime: This code defines a ‘ActTime' entity representing time delay action information, utilizing Lombok annotations for concise code. The entity includes fields for an ID (‘id’), routine ID (‘routineId’), timestamp (‘time’), and action order ('order1'), allowing for a simplified representation of these attributes.\\
                \item[-] TriAssistant: This code defines a ‘TriAssistant' entity representing assistant trigger information, utilizing Lombok annotations for concise code. The entity includes fields for an ID (‘id’), and command (‘command’), allowing for a simplified representation of these attributes.\\
                \item[-] TriLocation: This code defines a ‘TriLocation' entity representing location trigger information, utilizing Lombok annotations for concise code. The entity includes fields for an ID (‘id’), user action (‘mode’), longitude (‘longitude’), and latitude (‘latitude’), allowing for a simplified representation of these attributes.\\
                \item[-] TriPosture: This code defines a ‘TriPosture' entity representing posture trigger information, utilizing Lombok annotations for concise code. The entity includes fields for an ID (‘id’), user action (‘mode’), left top-X (‘leftTopX’), left top-Y (‘leftTopY’), right bottom-X ('rightBottomX'), right bottom-Y ('rightBottomY') and IP address (‘ip’), allowing for a simplified representation of these attributes. \\
                \item[-] TriTime: This code defines a ‘TriTime' entity representing time trigger information, utilizing Lombok annotations for concise code. The entity includes fields for an ID (‘id’), and timestamp (‘time’), allowing for a simplified representation of these attributes.\\
                \item[-] Device: This code defines a ‘Device' entity representing device information, utilizing Lombok annotations for concise code. The entity includes fields for an ID (‘id’), name (‘name’), device type (‘type’), user ID (‘userId’), switch status (‘switchStatus’), and text (‘text’), allowing for a simplified representation of these attributes.\\
                \item[-] Log: This code defines a ‘Log' entity representing log information, utilizing Lombok annotations for concise code. The entity includes fields for an ID (‘id’), user ID (‘userId’), timestamp (‘time’), and routine ID (‘routineId’), allowing for a simplified representation of these attributes. \\
                \item[-] Routine: This code defines a ‘Routine' entity representing routine information, utilizing Lombok annotations for concise code. The entity includes fields for an ID (‘id’), user ID (‘userId’), name (‘name’), switch status (‘onOff’), trigger type (‘’triggerType’) and trigger ID (‘triggerId’), allowing for a simplified representation of these attributes.\\
                \item[-] User: This code defines a ‘User' entity representing user information, utilizing Lombok annotations for concise code. The entity includes fields for an ID (‘id’), user ID (‘userId’), nickname (‘nickname’), password (‘password’), token (‘token’), and push notification setting (‘pushSetting’), allowing for a simplified representation of these attributes.\\
                \item[-] src/main/java/com.tempomate/pojo/result: This class provides a standardized method for encapsulating and conveying response results in a unified format across various parts of the application.\\
                \item[-] src/main/java/com.tempomate/service: This folder is responsible for handling business logic, coordinating data access, and performing various operations.\\
                \item[-] UserServiceImpl: This code defines the `UserServiceImpl` class, which implements the `UserService` interface. The class encapsulates the logic for user-related services, utilizing the `UserMapper` to interact with the database. Each method takes a `User` object as a parameter, performs the corresponding functionality, and returns the result.\\
                \item[-] UserService: This code defines the `UserService` interface, specifying methods for user-related operations such as user registration (`userSignup`), user login (`login`), changing a user's nickname (`change\_nickname`), and managing user push notification settings (`push\_setting`).\\
        \end{enumerate}
        % add database structures and apis specification
        % explain how you implemented the database and apis
            
        \item Computer vision\\
    
    \end{enumerate}

\end{enumerate}
\newpage
\section{\Large{Use cases}}

\begin{enumerate}[label=\arabic*]

\end{enumerate}
% VII.Software Installation Guide
% Not Required

% VIII.Discussion
% Write one or more paragraph(s) to describe any difficulty and experience you had. No less than 200 words. (e.g., communication difficulties in team, any non-technical difficulties, things to improve, and etc.) 

\clearpage
\section{\Large{Discussion}}

This project started when we heard the topic of matter and AI and wanted to create something new that doesn't exist yet. We are very proud of that with this project, we actually implemented a practical idea that could be seen somewhere in the future. \\

The limitations of this project are
\begin{enumerate}
    \item The MATTER protocol does not yet support camera devices, requiring a separate backend server.\\
    \item No hub is required to connect MATTER devices, but this means that the smartphone running the application must always be in the same network as the MATTER device. \\
    \item Unstablility and unreliability in recognizing human posture.\\
\end{enumerate}

While the first limitation is likely to be resolved over time and as versions of MATTER go up, we are concerned that the second limitation, the fact that you can currently only control your device from inside your home, and that you need a hub to control your MATTER device outside of your home, will be a barrier to entry for anyone using this platform. The third limitation is that we used an off-the-shelf model that was not well suited to what we were trying to do, and in an environment where we could not create thousands of datasets ourselves, so we trained it using a combination of datasets that were not well suited. In Next time, we look forward to tuning our own model to fit the project we're trying to do.\\

Also, the architecture is too complex for the size of the service. Due to the short development timeframe and limited manpower, we worked in parts, splitting the service into multiple backends. The main server uses Java SpringBoot, but OpenCV cannot be easily utilized in Java, which is one of the reasons why we separated the AI Processing Server using Python.\\

Since none of us had any experience with Android app development, we felt quite overwhelmed with this project, which was a combination of many things, not just making an app. Due to the short development period, we were not able to implement all the requirements, but it is quite encouraging that we were able to fully implement the UI/UX that we had planned, and the most important part of the project, connecting the device and setting the routine, and actually recognizing the human posture and starting the routine, worked completely. \\

However, we had a foreign student on our team, the language barrier was a bit of a challenge. Not everyone on the team was familiar with English, the common language, so we used Korean as much as possible, but used English or a translator to communicate when things didn't go well. During the development process, there were times when this communication problem led to some misunderstandings about what to do or what the project was about, and it was quite difficult to recognize and correct.\\

We felt a strong need for systematic documentation. While we all understood the concept in the planning stages, we all had different ideas about what features we wanted to implement and how we wanted to implement them, and it wasn't uncommon for us to end up with a deliverable that wasn't what someone expected and had to be reworked. We must not assume that anyone else knows what's going on in my head. Minimizing this confusion by documenting as much detail and clarity as possible, even if it's about a single button, will hopefully speed up the project and reduce the stress of working on it.\\

The process of planning, architecting, and designing a project in a somewhat structured way was refreshing to us, as it was quite a different process than the traditional haphazard development process. The project was very similar to the Waterfall process, and while I was able to see the benefits of using this process for development. I also learned that plans can always change and be pushed back due to various factors. Next time, I'd like to approach the creation and refinement of a minimum viable product through an agile methodology, and I'm hopeful that the trials and tribulations we went through this time will allow us to complete the project more smoothly next time.\\


\clearpage
\bibliographystyle{IEEEtran}
\bibliography{references}

\end{document}
